\documentclass[a5paper,12pt,ngerman,print,grid=front]{kartei}
\usepackage[german]{babel}
\usepackage[utf8x]{inputenc}
\usepackage{amsmath}
\usepackage{graphicx}
\usepackage[version=3]{mhchem}
\let\oldce\ce
\renewcommand*{\ce}[1]{\begingroup\color{blue}\oldce{#1}\endgroup}
%%%%%%%%%%%%%%%%%%%%%%%%%%%%%%%%%%%%%%%%%%%%%%%%%%%%%%%%%%%%%%%%%%%%%%%%%%
%%%%%%%%%%%%%%%%%%%%%%%%%%%%%%%%%%%%%%%%%%%%%%%%%%%%%%%%%%%%%%%%%%%%%%%%%%


\title{Prüfungsfragen Molekül- und Festkörperphysik}
%%%%%%%%%%%%%%%%%%%%%%%%%%%%%%%%%%%%%%%%%%%%%%%%%%%%%%%%%%%%%%%%%%%%%%%%%%


\begin{document}


%%%%%%%%%%%%%%%%%%%%%%%%%%%%%%%%%%%%%%%%%%%%%%%%%%%%%%%%%%%%%%%%%%%%%%%%%%
%%%%%%%%%%%%%%%%%%%%%%%%%%%%%%%%%%%%%%%%%%%%%%%%%%%%%%%%%%%%%%%%%%%%%%%%%%
\section*{Molekülphysik}
%%%%%%%%%%%%%%%%%%%%%%%%%%%%%%%%%%%%%%%%%%%%%%%%%%%%%%%%%%%%%%%%%%%%%%%%%%
%%%%%%%%%%%%%%%%%%%%%%%%%%%%%%%%%%%%%%%%%%%%%%%%%%%%%%%%%%%%%%%%%%%%%%%%%%



%%%%%%%%%%%%%%%%%%%%%%%%%%%%%%%%%%%%%%%%%%%%%%%%%%%%%%%%%%%%%%%%%%%%%%%%%%
	\begin{karte}{
    	Skizzieren Sie die symmetrische und antisymmetrische Wellenfunktion eines 
        homonuklearen zweiatomigen Moleküls. (anti/bindende Wellenfkt.)
        }
		$\Psi_s=\frac{1}{\sqrt{2+2S_{AB}}}(\Phi_A+\Phi_B)$ \\
        
		$\Psi_a=\frac{1}{\sqrt{2+2S_{AB}}}(\Phi_A-\Phi_B)$ \\
        
		Mit "Überlappungsintegral": $S_{AB}=Re \int \Phi_A \Phi_B$ \\
        
		Wellenfunktion Kern A: $ \Phi_A= \frac{1}{\sqrt{a_0^3}}e^{-\frac{r_A}{a_0}}$ \\
        
        Bohrscher Atomradius: $a_0$ \\
        
        $\Psi_s$ führt zu bindenden, $\Psi_a$ zu antibindenden Zustand
        
    \end{karte}


%%%%%%%%%%%%%%%%%%%%%%%%%%%%%%%%%%%%%%%%%%%%%%%%%%%%%%%%%%%%%%%%%%%%%%%%%%
    \begin{karte}{
        Welche Überlegungen führen zur MO- und VB-Näherung?
        }
        	
        Da das $\ce{H2}$ Molekül 2 Elektronen besitzt und man diese Wechselwirkung zwischen 
        den Elektronen natürlich berücksichtigen muss, führt dies zu einer nicht mehr 
        analytisch lösbaren Schrödingergleichung. 
        (Separierung der SG wie beim $\ce{He^+}$ Molekül nicht mehr möglich) \\
		Man muss also Näherungsmethoden finden.
        
        \begin{itemize}
			\item Molekül-Orbital Näherung \\
            
        	Die Überlegung ist, dass das Molekül für Atomabstand $R \rightarrow \infty$ 
        	in 2 $\ce{H}$ Atome im 1s Zustand zerfällt. 
        	Man kann nun also einen Ansatz mit $\ce{H}$ 1s Wellenfunktionen machen und für 
        	das ganze Molekül einen Produktansatz aus 2 dieser Funktionen 
        	(für Atom A und Atom B) bilden.
        	
        	$\Psi_{\vec{r_1},\vec{r_2}} = \Psi_s(\vec{r_1})\Psi_s(\vec{r_2})$ symmetrisch bezüglich
        	Vertauschung der Elektronen, aber aus dem Pauli Prinzip
            folgt, dass die Gesamtwellenfunktion antisymmetrisch sein muss, also muss es 
            einen antisymmetrischen Spinanteil in der Gesamtwellenfunktion geben:
            
            $\Psi(\vec{r_1},\vec{r_2},\vec{s_1},\vec{s_2})=
            \Psi_s(\vec{r_1})\Psi_s(\vec{r_2})(\chi^+(1)\chi^-(2) - \chi^+(2)\chi^-(1))$
            
            Wobei man schreiben kann: 
            $\chi^+ = \alpha$, $\chi^- = \beta$ 
            (wobei $\alpha(1)$: $m_s(1) = +\frac{1}{2}$, $\beta(1): m_s(2) = -\frac{1}{2})$
            
            Wenn man nun noch die symmetrische Wellenfunktion 
            $\chi^+ = \alpha$, $\chi^- = \beta$ 
            (wobei 
            $\alpha(1): m_s(1)= +\frac{1}{2}$, 
            $\beta(1):  m_s(2)= -\frac{1}{2}$)
            
            Wenn mam nun die symmetrische Wellenfunktion 
            $\Psi= \frac{1}{\sqrt{2+2S_{AB}}}(\Phi_A+\Phi_B)$
            für die $\Psi_S(\vec{r_1} ) \Psi_S(\vec{r_2} )$ einsetzt und die Abkürzungen $\Phi_A=a$ und 
            $\Phi_B=b$ einführt bekommt man für den räumlichen Anteil:
            $  \Psi( \vec{r_1}, \vec{r_2} ) = \frac{1}{2+2S_{AB}} \left[ a(1)+b(1) \right]\left[ a(2)+b(2) \right]   $
            
        \item Valenzbindungs Näherung \\
             
            
		\end{itemize}
        
    \end{karte}
        
        
%%%%%%%%%%%%%%%%%%%%%%%%%%%%%%%%%%%%%%%%%%%%%%%%%%%%%%%%%%%%%%%%%%%%%%%%%%
    \begin{karte}{
        Vergleiche Molekülorbitale und Valenzbindungsmodelle (Heitler London Näherung).
        }
        
        \begin{itemize}
			\item Molekül-Orbital Näherung \\
			
            Die Idee ist, dass das Molekül für Atomabstand $R \rightarrow \infty$ in 2 
            $\ce{H}$ Atome im 1s Zustand zerfällt. 
            Man muss nun also einen Ansatz mit $\ce{H}$ 1s Wellenfunktionen machen und für 
            das ganze Molekül einen Produktansatz aus 2 dieser Funktionen 
            (für Atom A und Atom B) bilden. \\
            Mit Beachtung des Pauli-Prinzips folgt dann:
            $$\Psi_{\overrightarrow{r_1},\overrightarrow{r_2}} = 
            \frac{1}{2+2S_{AB}}[a(1)+b(1)][a(2)+b(2)]$$
            
            \item Valenzbindungs Näherung \\
             
             Bei der VB geht man von Molekükorbitalen aus, also setzt man direkt eine Wellenfunktion für 2 Elektronen an. Durch die Forderung, dass die Wellenfunktion entweder symmetrisch oder antisymmetrisch ist, folgt der Ansatz einer Linearkombination beider möglichen Lösungen.
             $ \Psi_1 = c_1\phi_A(1)\phi_B(2) $ und $ \Psi_2 = c_1\phi_A(2)\phi_B(1) $ zu
             $$  \Psi_{S,A} = c \left[   \phi_A(1) \phi_B(2) \pm   \phi_A(2) \phi_B(1)   \right]  $$
             
             Wenn man die Klammern in der MO Gleichung ausmultipliziert, sieht man, dass im VB Ansatz die Terme, die beschreiben, dass sich beide Elektronen entweder an Kern A oder Kern B befinden, fehlen.
             Jetzt ist dieser Zusand zwar unwahrscheinlicher als die anderen Zustände, wird aber in der MO mit gleicher Stärke berücksichtigt. Die VB berücksichtigt dafür diese Terme gar nicht.
             Ansonsten sind die Näherungen gleich.
            
		\end{itemize}
        
    \end{karte}
    
    
%%%%%%%%%%%%%%%%%%%%%%%%%%%%%%%%%%%%%%%%%%%%%%%%%%%%%%%%%%%%%%%%%%%%%%%%%%
    \begin{karte}{
        Erläutere die Verbesserung des MO-Ansatzes.
        }
    \end{karte}
    
    
%%%%%%%%%%%%%%%%%%%%%%%%%%%%%%%%%%%%%%%%%%%%%%%%%%%%%%%%%%%%%%%%%%%%%%%%%%
    \begin{karte}{
        Warum ist $\ce{He2+}$ stabil aber $\ce{He2}$ nicht?
        }
        
        Beim $\ce{He2}$ werden die $2e$ im $1s \sigma_g$ Orbital (bindend) praktisch komplett durch die zwei $e^-$ im antibindenden $1s\sigma_u$ Orbital kompensiert
                
    \end{karte}

  
%%%%%%%%%%%%%%%%%%%%%%%%%%%%%%%%%%%%%%%%%%%%%%%%%%%%%%%%%%%%%%%%%%%%%%%%%%
    \begin{karte}{
        Warum ist das $\ce{H2}$ Molekül, nicht aber das $\ce{He2}$ Molekül stabil?
        }
    \end{karte}
  
  
%%%%%%%%%%%%%%%%%%%%%%%%%%%%%%%%%%%%%%%%%%%%%%%%%%%%%%%%%%%%%%%%%%%%%%%%%%
    \begin{karte}{
        Diskutieren Sie die Bindung im $\ce{Na2}$ Molekül.
        }
    \end{karte}
  
  
%%%%%%%%%%%%%%%%%%%%%%%%%%%%%%%%%%%%%%%%%%%%%%%%%%%%%%%%%%%%%%%%%%%%%%%%%%
    \begin{karte}{
        Diskutieren Sie die Bindung im $\ce{Li2}$ und $\ce{CH4}$ Molekül.
        }
    \end{karte}
  
  
%%%%%%%%%%%%%%%%%%%%%%%%%%%%%%%%%%%%%%%%%%%%%%%%%%%%%%%%%%%%%%%%%%%%%%%%%%
    \begin{karte}{
        Diskutiere die chemische Bindung im $\ce{B2}$ Modell.
        }
    \end{karte}
  
  
%%%%%%%%%%%%%%%%%%%%%%%%%%%%%%%%%%%%%%%%%%%%%%%%%%%%%%%%%%%%%%%%%%%%%%%%%%
    \begin{karte}{
        Molekülbindungsarten? (kovalente, ionische, Valenzbindung)
        }
    \end{karte}
  
  
%%%%%%%%%%%%%%%%%%%%%%%%%%%%%%%%%%%%%%%%%%%%%%%%%%%%%%%%%%%%%%%%%%%%%%%%%%
	\begin{karte}{
		Beschreiben sie die Elemente der Born-Openheimer Näherung.
        }
          
          
          
    \end{karte}


%%%%%%%%%%%%%%%%%%%%%%%%%%%%%%%%%%%%%%%%%%%%%%%%%%%%%%%%%%%%%%%%%%%%%%%%%%
	\begin{karte}{
		Beschreibe die Rotation zweiatomiger Moleküle.
        }
          
          
          
    \end{karte}


%%%%%%%%%%%%%%%%%%%%%%%%%%%%%%%%%%%%%%%%%%%%%%%%%%%%%%%%%%%%%%%%%%%%%%%%%%
	\begin{karte}{
		Vorteile Morse-Potential gegnüber harmonischem Potential.
        }
          
          
          
    \end{karte}


%%%%%%%%%%%%%%%%%%%%%%%%%%%%%%%%%%%%%%%%%%%%%%%%%%%%%%%%%%%%%%%%%%%%%%%%%%
	\begin{karte}{
		Wovon hängt die Intensität einer Spektrallinie ab?
        }
          
          
          
    \end{karte}


%%%%%%%%%%%%%%%%%%%%%%%%%%%%%%%%%%%%%%%%%%%%%%%%%%%%%%%%%%%%%%%%%%%%%%%%%%
	\begin{karte}{
		Diskutiere das Schwingungs-Rotations-Spektrum bei zweiatomigen Molekülen.
        }
          
          
          
    \end{karte}


%%%%%%%%%%%%%%%%%%%%%%%%%%%%%%%%%%%%%%%%%%%%%%%%%%%%%%%%%%%%%%%%%%%%%%%%%%
	\begin{karte}{
		Wovon hängt die Intensität bei einmen elektronischen Übergang ab?
        }
          
          
          
    \end{karte}


%%%%%%%%%%%%%%%%%%%%%%%%%%%%%%%%%%%%%%%%%%%%%%%%%%%%%%%%%%%%%%%%%%%%%%%%%%
	\begin{karte}{
		Franck-Condon-Prinzip und Intensität des Schwingungsüberganges.
        }
          
          
          
    \end{karte}


%%%%%%%%%%%%%%%%%%%%%%%%%%%%%%%%%%%%%%%%%%%%%%%%%%%%%%%%%%%%%%%%%%%%%%%%%%
	\begin{karte}{
		Diskutiere den Aufbau eines $\ce{H2O}$ Molekül.
        }
          
          
          
    \end{karte}


%%%%%%%%%%%%%%%%%%%%%%%%%%%%%%%%%%%%%%%%%%%%%%%%%%%%%%%%%%%%%%%%%%%%%%%%%%
	\begin{karte}{
		Diskutiere den Aufbau eines $\ce{CH4}$ Moleküls im Rahmen der Molekularorbitalnäherung.
        }
          
          
          
    \end{karte}


%%%%%%%%%%%%%%%%%%%%%%%%%%%%%%%%%%%%%%%%%%%%%%%%%%%%%%%%%%%%%%%%%%%%%%%%%%
	\begin{karte}{
		Beschreibe das $\ce{NH3}$ Molekül.
        }
          
          
          
    \end{karte}


%%%%%%%%%%%%%%%%%%%%%%%%%%%%%%%%%%%%%%%%%%%%%%%%%%%%%%%%%%%%%%%%%%%%%%%%%%
	\begin{karte}{
		Erklären Sie die Hybridisierung anhand des Benzol-Molekül.
        }
          
          
          
    \end{karte}


%%%%%%%%%%%%%%%%%%%%%%%%%%%%%%%%%%%%%%%%%%%%%%%%%%%%%%%%%%%%%%%%%%%%%%%%%%
	\begin{karte}{
		Was sind die Normalkoordinaten bei der Beschreibung der Schwingungen mehratomiger Moleküle?
        }
          
          
          
    \end{karte}




%%%%%%%%%%%%%%%%%%%%%%%%%%%%%%%%%%%%%%%%%%%%%%%%%%%%%%%%%%%%%%%%%%%%%%%%%%
%%%%%%%%%%%%%%%%%%%%%%%%%%%%%%%%%%%%%%%%%%%%%%%%%%%%%%%%%%%%%%%%%%%%%%%%%%
\section*{Festkörperphysik}
%%%%%%%%%%%%%%%%%%%%%%%%%%%%%%%%%%%%%%%%%%%%%%%%%%%%%%%%%%%%%%%%%%%%%%%%%%
%%%%%%%%%%%%%%%%%%%%%%%%%%%%%%%%%%%%%%%%%%%%%%%%%%%%%%%%%%%%%%%%%%%%%%%%%%



%%%%%%%%%%%%%%%%%%%%%%%%%%%%%%%%%%%%%%%%%%%%%%%%%%%%%%%%%%%%%%%%%%%%%%%%%%
	\begin{karte}{
		Festkörperbindungsarten? (Diskutiere die metallische Bindung.)
        }
          
          
          
    \end{karte}


%%%%%%%%%%%%%%%%%%%%%%%%%%%%%%%%%%%%%%%%%%%%%%%%%%%%%%%%%%%%%%%%%%%%%%%%%%
	\begin{karte}{
		Graphit versus Diamant: Wodurch unterscheidet sich die chemische Bindung in diesen
        beiden Festkörpern?
		}
          
          
          
    \end{karte}


%%%%%%%%%%%%%%%%%%%%%%%%%%%%%%%%%%%%%%%%%%%%%%%%%%%%%%%%%%%%%%%%%%%%%%%%%%
	\begin{karte}{
		Unterschied zwischen fcc und hcp?
		}
          
          
          
    \end{karte}


%%%%%%%%%%%%%%%%%%%%%%%%%%%%%%%%%%%%%%%%%%%%%%%%%%%%%%%%%%%%%%%%%%%%%%%%%%
	\begin{karte}{
		Vergleiche bcc- und fcc-Gitter.
		}
          
          
          
    \end{karte}


%%%%%%%%%%%%%%%%%%%%%%%%%%%%%%%%%%%%%%%%%%%%%%%%%%%%%%%%%%%%%%%%%%%%%%%%%%
	\begin{karte}{
		Unterschied starres / zeitlich veränderliches Gitter.
		}
          
          
          
    \end{karte}


%%%%%%%%%%%%%%%%%%%%%%%%%%%%%%%%%%%%%%%%%%%%%%%%%%%%%%%%%%%%%%%%%%%%%%%%%%
	\begin{karte}{
		Definition und Vorteile des reziproken Gitters.
		}
          
          
          
    \end{karte}


%%%%%%%%%%%%%%%%%%%%%%%%%%%%%%%%%%%%%%%%%%%%%%%%%%%%%%%%%%%%%%%%%%%%%%%%%%
	\begin{karte}{
		Erkäre die Laue-Bedingung mit \\
        a) Ewaldschen Konstruktion \\
        b) Bragg-Reflexion.
		}
          
          
          
    \end{karte}


%%%%%%%%%%%%%%%%%%%%%%%%%%%%%%%%%%%%%%%%%%%%%%%%%%%%%%%%%%%%%%%%%%%%%%%%%%
	\begin{karte}{
		Skizziere die Brillouinschen-Zonen eines Parallelogrammgitters.
		}
          
          
          
    \end{karte}


%%%%%%%%%%%%%%%%%%%%%%%%%%%%%%%%%%%%%%%%%%%%%%%%%%%%%%%%%%%%%%%%%%%%%%%%%%
	\begin{karte}{
		Skizziere die Brillouinsche Zone eines 2D hexagonalen Gitters.
		}
          
          
          
    \end{karte}


%%%%%%%%%%%%%%%%%%%%%%%%%%%%%%%%%%%%%%%%%%%%%%%%%%%%%%%%%%%%%%%%%%%%%%%%%%
	\begin{karte}{
		Strukturfaktor / Atomfaktor.
		}
          
          
          
    \end{karte}


%%%%%%%%%%%%%%%%%%%%%%%%%%%%%%%%%%%%%%%%%%%%%%%%%%%%%%%%%%%%%%%%%%%%%%%%%%
	\begin{karte}{
		Unterschiede im Strukturfaktor im einfachen kubischen und im bcc-Gitter?
		}
          
          
          
    \end{karte}


%%%%%%%%%%%%%%%%%%%%%%%%%%%%%%%%%%%%%%%%%%%%%%%%%%%%%%%%%%%%%%%%%%%%%%%%%%
	\begin{karte}{
		Was führt zum Auslöschen bestimmter Reflexe in eimen bcc?
		}
          
          
          
    \end{karte}


%%%%%%%%%%%%%%%%%%%%%%%%%%%%%%%%%%%%%%%%%%%%%%%%%%%%%%%%%%%%%%%%%%%%%%%%%%
	\begin{karte}{
		Dispersionsrelation einer zweiatomigen linearen Kette. (Phononendispersion)
		}
          
          
          
    \end{karte}


%%%%%%%%%%%%%%%%%%%%%%%%%%%%%%%%%%%%%%%%%%%%%%%%%%%%%%%%%%%%%%%%%%%%%%%%%%
	\begin{karte}{
		Unterschied akustische und optische Phononen.
		}
          
          
          
    \end{karte}


%%%%%%%%%%%%%%%%%%%%%%%%%%%%%%%%%%%%%%%%%%%%%%%%%%%%%%%%%%%%%%%%%%%%%%%%%%
	\begin{karte}{
		Wodurch unterscheiden sich die Dispersionsrelationen für Phononen im primitiv 
        kubischen Gitter und der $\ce{CsCl}$ Struktur (bcc kub. raumzentriert.)?
		}
          
          
          
    \end{karte}


%%%%%%%%%%%%%%%%%%%%%%%%%%%%%%%%%%%%%%%%%%%%%%%%%%%%%%%%%%%%%%%%%%%%%%%%%%
	\begin{karte}{
		Zustandsdichte vom freien Elektronengases?
		}
          
          
          
    \end{karte}


%%%%%%%%%%%%%%%%%%%%%%%%%%%%%%%%%%%%%%%%%%%%%%%%%%%%%%%%%%%%%%%%%%%%%%%%%%
	\begin{karte}{
		Fermi-Verteilung für T = 0K und bei endlichen Temperaturen.
		}
          
          
          
    \end{karte}


%%%%%%%%%%%%%%%%%%%%%%%%%%%%%%%%%%%%%%%%%%%%%%%%%%%%%%%%%%%%%%%%%%%%%%%%%%
	\begin{karte}{
		Entstehung elektronischer Bänder im Festkörper.
		}
          
          
          
    \end{karte}


%%%%%%%%%%%%%%%%%%%%%%%%%%%%%%%%%%%%%%%%%%%%%%%%%%%%%%%%%%%%%%%%%%%%%%%%%%
	\begin{karte}{
		Bildung elektronischer Bänder mit der Methode des ”stark gebundenen“ Elektrons. 
        (”tight binding“ Näherung)
		}
          
          
          
    \end{karte}


%%%%%%%%%%%%%%%%%%%%%%%%%%%%%%%%%%%%%%%%%%%%%%%%%%%%%%%%%%%%%%%%%%%%%%%%%%
	\begin{karte}{
		Wovon hängt die Breite von elektronischen Bändern ab?
		}
          
          
          
    \end{karte}


%%%%%%%%%%%%%%%%%%%%%%%%%%%%%%%%%%%%%%%%%%%%%%%%%%%%%%%%%%%%%%%%%%%%%%%%%%
	\begin{karte}{
		Wann ist ein Festkörper Leiter bzw. Nichtleiter?
		}
          
          
          
    \end{karte}

%%%%%%%%%%%%%%%%%%%%%%%%%%%%%%%%%%%%%%%%%%%%%%%%%%%%%%%%%%%%%%%%%%%%%%%%%%


\end{document}

